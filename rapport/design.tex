\section{Design and implementation}
In this section the considerations for the design will be discussed. This
includes the overall design for the application as well as the problems that the
individual features posed. The section also details how the proposed solutions
were carried out and implemented in the application.

\subsection{Overall design}
%Jonas
Mvvm, wpf, data bindings
Model assembly
\subsection{User interface}
Even though that user interface design was not a main scope of the project, some
thought is required as it greatly impacts the user experience. 

The overall decision to make it wether to whether to use an existing design
framework, or to create a whole new layout with toolbars, menues etc. Since it
is not in the scope of the project and the benefit would be minimal it was decided
to go with an already tested and know interface. 

The traditional Windows application is defined by its layered menus and
toobars,
which execls at more complicated programs where submenus are essential.

The other viable alternative is the \textit{Ribbon}\footnote{http://msdn.microsoft.com/en-us/library/windows/desktop/dd316910(v=vs.85).aspx} framework which is know for
its big toolbar with the most used functions ready for use.

Based on the original program mockup as previously seen in figure \ref{mockup},
it was clear that all the menu items could be in such a bar that the Ribbon
interface offers without creating additional layers.

It also seemed a good fit for a graphical editing tool, as the Ribbon interface
has an emphasis on having a graphical display for each action. This makes the
tool a lot easier to use for new users, as for example a relation could be
accompanied by a graphical representation showing exactly how the relation will
look like.

\subsection{Klasses and relations}
%Jonas
Relations so extendable

\subsection{Saving \& loading}
%Peter
Serialization, threading..

\subsection{Undo / Redo}
%Peter
Commands, scope (what can we undo / redo)
\subsection{Image export}
Why png? Cropping
