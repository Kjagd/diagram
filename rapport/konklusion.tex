\section{Conclusion}

This document has explained how a diagram tool for the windows platform was
created with \texttt{WPF}. 

The program was built around the \textit{Model View ViewModel} design pattern
which enabled us to keep the business and presentation layer seperated. The
business layer was implemented as a \texttt{model} assembly, and the
presentation layer was written in \texttt{XAML}. By using data bindings the two
were connected in a powerful way. 

During the project we had to set a scope that fit the available time. Since none
of the team-members had worked with the \texttt{WPF} framework previously, a lot
of time went towards learning it in the beginning. In this period mostly basic
features such as the \texttt{Ribbon} toolbar and the class model and view was
created. When the team got more proficient with tools, more advanced features
were added, such as a flexible way to add relations, saving and loading diagrams
and exporting to png. 

In the end the product turned out to be a rather complete prototype as all of
the basic functionality was implemented and even a few of the advanced features
was introduced. The program lays a good foundation for further development, as
it would be easy to extend with additional diagram types, canvas controls, etc.
