\section{Tests}

The tests made for this project is primarily done with unit
tests. The graphical user interface has been tested by hand.

\subsection{Unit Test 1 - Add new class}

The user may want to add a new class to the model. By pressing the button, it is
possible to observe that a new class indeed appears. The test ensures that a
new \texttt{Klass} instance has been added to the \texttt{MainViewModel}'s
collection.

\subsection{Unit Test 2 - Remove class}

The user may also want to remove a class from the diagram. In this case, it is
necessary to first select a class, and then call remove.

It is then asserted that the class is removed from the \texttt{Klasses}
collection.

It is also tested for the case when no class is selected. In this case, nothing
should happen.

Instead of letting nothing happen, it would be possible to throw an exception.
This would also make it possible to warn the user if necessary.

\subsection{Unit Test 3 - Copy/Paste class}

This test is to make sure that a copy/pasted class is added to the ViewModel,
and that it is a deep copy.

This functionality also relies on a selected class, so it is also tested that
nothing happens when no class is selected.

The user may also want to paste a class multiple times. These extra copies does
also have to be deep copies, and the user should be able to paste multiple times
without having to reselect a class.


\subsection{Unit Test 4 - Setting a relation}

The user should be able to add a relation between 2 classes. It is crucial for
the model, that a relation has a reference to each class, and that each class 
has a reference to each relation.

The test ensures that the above requirements are satisfied.

\subsection{Unit Test 5 - Unsetting a relation}

It is possible to remove a relation. Since this affects the state of the program
multiple places when set, they all have to be reverted when unset.


\subsection{Code coverage}
The tests cover approximately $30\%$ of all the code. Achieving $100\%$ would
likely take as long time as making the project itself, so only the most
important parts, primarily the model, was tested. Testing the model ensures 
that the \emph{machinery} works. Testing the GUI is easier done by hand.

It would also be difficult to test the GUI part of the program, and since e.g.
AddRelation relies on mouse clicks, it would be hard to make unit test for.

\begin{figure}[H]
\centering
\includegraphics[width=0.85\linewidth]{img/coverage}
\caption{Screenshot of the Visual Studio code coverage tools result.}
\end{figure}

