\section{Discussion}
In this section the resulting product is discussed including what gave us trouble and
which features and changes could be relevant in the future if further
development was needed.

\subsection{Problems and bugs}
A problem we had for a long time was how focus was handled. For example if the
user is attempting to drag a class by clicking on a textfield it can cause some
havoc since it is unclear if the user wants to drag the class or edit the text. 

Another problem focus gave us, was how to loose focus again when for example the
user clicks on the background. This was sounds easy, but it turned out that a
workaround had to be used, where the entire canvas is wrapped in a
\texttt{StackPanel} with transparent background before it could be made
focusable.

Another problem we had was providing undo/redo functionality for all functions,
primarily since it was time consuming to create this for all actions. This meant
that we at some point had some bugs with deleting classes, as the redo function
would not correctly add the relations back. This was later fixed, but more
undo/redo functions such as editing text would be more desireble. 

\subsection{Futher development}
Even though the program is largely useable, more features could always be added.
If we had more time, some of our next features/improvements would be:

\begin{itemize}
		\item More relations\\As it it very easy to add more relations this
				could be done very fast, and enable the program to be used to
				create a wider range of diagrams.

		\item Better canvas controls\\Right now the canvas is very static, but
				for larger diagrams it would be nice to be able to pan around
				and zoom in on parts of the diagram.

		\item Strict UML\\As it is now only a subset of the UML syntax is
				supported. Support for field visibility, abstract classes etc
				would be required to create more formal diagrams.
		\item New Diagram types\\Right now only class diagrams are supported,
				but UML also covers sequence diagrams and state machines. This
				could be implemented reusing common functionality such as
				drawing arrows between nodes. Certain tasks would require new
				actions, and the toolbar would have to be extended.
\end{itemize}
